\documentclass{ltxdoc}
\usepackage{url}

\begin{document}

\changes{0.1}{2013/10/20}{Initial version}

\title{The \textsf{beamerpacing} package}
\author{Adam Crume \\ \url{adamcrume@gmail.com}}
\date{0.1 from 2013/10/20}

\maketitle

\section{Introduction}

\texttt{beamerpacing} is used to provide an estimated end time for each slide in a presentation.
When displayed on an alternate screen, this is useful for keeping an appropriate pacing.

\section{Usage}

\DescribeMacro{\slideweight}
Sets the weight for the current slide.
The default weight is 1.
Weights are typically set like so:
\begin{verbatim}\begin{frame}
  Slide content here
  \slideweight{.75}
\end{frame}\end{verbatim}
The exact position of \verb|slideweight| inside the frame is not important.

The weight is per \emph{slide}, not frame, so this can be set separately for each overlay.
A frame with overlays may set the weights as follows:
\begin{verbatim}\begin{frame}
  \begin{itemize}
    \item This frame
    \item<2-> has overlays.
  \end{itemize}
  \only<1>{\slideweight{1.5}}
  \only<2>{\slideweight{.75}}
\end{frame}\end{verbatim}
Note that using \verb|\slideweight| after \verb|\pause| or within \verb|\item<2->| currently does not work as expected.

\DescribeMacro{\slidefinishtime}
Inserts the estimated end time of the current slide, in \texttt{mm:ss} format.
This will usually be called from a \texttt{note page} template.

\DescribeMacro{slide with time}
This is a simple \texttt{note page} template that overlays the estimated end time on top of the slide contents.

\end{document}
